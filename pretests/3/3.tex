\documentclass[12pt,addpoints]{exam}
\usepackage[T2A]{fontenc}
\usepackage[utf8]{inputenc}
\usepackage[russian]{babel}

\usepackage[margin=1in]{geometry}
\usepackage{amsmath,amssymb}
\usepackage{multicol}
\usepackage{tikz}

\newcommand{\class}{Оптимизация}
\newcommand{\term}{Осенний семестр 2017- 2018}
\newcommand{\examnum}{Тест 3}
\newcommand{\examdate}{1/1/2017}
\newcommand{\timelimit}{60 Minutes}

\pagestyle{head}
\firstpageheader{}{}{}
\runningheader{\class}{\examnum\ - Стр \thepage\ из \numpages}{\examdate}
\runningheadrule

\def\width{16}
\def\hauteur{7}
\pointpoints{балл}{баллов}
\hpword{Баллов:}
\hqword{Задача}
\htword{Итог}
\hsword{Оценка:}
\vpword{Баллов:}
\vqword{Задача}
\vtword{Итог}
\vsword{Оценка:}

\begin{document}
	
	\noindent
	\begin{tabular*}{\textwidth}{l @{\extracolsep{\fill}} r @{\extracolsep{6pt}} l}
		\textbf{\class} & \textbf{ФИО:} & \makebox[3in]{\hrulefill}\\
		\textbf{\term} &&\\
		\textbf{\examnum} &&
	\end{tabular*}\\
	\rule[2ex]{\textwidth}{2pt}
	
	
	
	\begin{questions}
		\question[1] Что такое выпуклый конус?
		\fillwithdottedlines{4em}
		\question[1] Как определить афинную оболочку множества $S$?
		\fillwithdottedlines{4em}
		\question[1] Какое множество называется выпуклым?
		\fillwithdottedlines{4em}
		\question[2] Докажите, что полупространство (т.е. множество $\{ \mathbf{x} \mid \mathbf{a}^{T} \mathbf{x} \leq c \}$) - выпуклое множество. 
		\fillwithdottedlines{6em}
		\question[2] Докажите, что шар в $\mathbb{R}^n$ (т.е. множество $\{ \mathbf{x} \mid \| \mathbf{x} - \mathbf{x}_c \| \leq r \}$) - выпуклый. 
		\fillwithdottedlines{6em}
		\question[3] Доказать, что если $S$ - выпукло, то $S+S = 2S$. Привести контрпример в случае, когда $S$ - не выпукло.
		\fillwithdottedlines{8em}
		
		\begin{center}
			\tiny
			\gradetable[h]
		\end{center}
	\end{questions}



\end{document}
