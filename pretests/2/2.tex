\documentclass[12pt,addpoints]{exam}
\usepackage[T2A]{fontenc}
\usepackage[utf8]{inputenc}
\usepackage[russian]{babel}

\usepackage[margin=1in]{geometry}
\usepackage{amsmath,amssymb}
\usepackage{multicol}
\usepackage{tikz}

\newcommand{\class}{Оптимизация}
\newcommand{\term}{Осенний семестр 2017- 2018}
\newcommand{\examnum}{Тест 2}
\newcommand{\examdate}{1/1/2014}
\newcommand{\timelimit}{60 Minutes}

\pagestyle{head}
\firstpageheader{}{}{}
\runningheader{\class}{\examnum\ - Стр \thepage\ из \numpages}{\examdate}
\runningheadrule

\def\width{16}
\def\hauteur{7}
\pointpoints{балл}{баллов}
\hpword{Баллов:}
\hqword{Задача}
\htword{Итог}
\hsword{Оценка:}
\vpword{Баллов:}
\vqword{Задача}
\vtword{Итог}
\vsword{Оценка:}

\begin{document}
	
	\noindent
	\begin{tabular*}{\textwidth}{l @{\extracolsep{\fill}} r @{\extracolsep{6pt}} l}
		\textbf{\class} & \textbf{ФИО:} & \makebox[3in]{\hrulefill}\\
		\textbf{\term} &&\\
		\textbf{\examnum} &&
	\end{tabular*}\\
	\rule[2ex]{\textwidth}{2pt}
	
	
	
	\begin{questions}
		\question[5] Используя геометрические интерпретации, решить экстремальную задачу: $extr (x_1^2 + x_2^2)$ при условии: \{$x: x_1 x_2 \le 0; |x_2 - x_1| \le 2$\}
		
		\begin{center}
			\begin{tikzpicture}[x=1cm, y=1cm, semitransparent]
			\draw[step=1mm, line width=0.1mm, black!30!white] (0,0) grid (\width,\hauteur);
			\draw[step=5mm, line width=0.2mm, black!40!white] (0,0) grid (\width,\hauteur);
			\draw[step=5cm, line width=0.5mm, black!50!white] (0,0) grid (\width,\hauteur);
			\draw[step=1cm, line width=0.3mm, black!90!white] (0,0) grid (\width,\hauteur);
			\end{tikzpicture}
		\end{center}
		
		\question[5] Используя геометрические интерпретации, решить экстремальную задачу: $extr (x_1 + 2x_2)$ при условии: \{$x: x_1^2 + x_2^2 \le 1; x_1 - x_2 \le 0 $\}
		
		\begin{center}
			\begin{tikzpicture}[x=1cm, y=1cm, semitransparent]
			\draw[step=1mm, line width=0.1mm, black!30!white] (0,0) grid (\width,\hauteur);
			\draw[step=5mm, line width=0.2mm, black!40!white] (0,0) grid (\width,\hauteur);
			\draw[step=5cm, line width=0.5mm, black!50!white] (0,0) grid (\width,\hauteur);
			\draw[step=1cm, line width=0.3mm, black!90!white] (0,0) grid (\width,\hauteur);
			\end{tikzpicture}
		\end{center}
		
		\begin{center}
			\gradetable[h]
		\end{center}
	\end{questions}



\end{document}
