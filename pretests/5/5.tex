\documentclass[12pt,addpoints]{exam}
\usepackage[T2A]{fontenc}
\usepackage[utf8]{inputenc}
\usepackage[russian]{babel}

\usepackage[margin=1in]{geometry}
\usepackage{amsmath,amssymb}
\usepackage{multicol}
\usepackage{tikz}

\newcommand{\class}{Оптимизация}
\newcommand{\term}{Осенний семестр 2017- 2018}
\newcommand{\examnum}{Тест 5}
\newcommand{\examdate}{1/1/2017}
\newcommand{\timelimit}{60 Minutes}

\pagestyle{head}
\firstpageheader{}{}{}
\runningheader{\class}{\examnum\ - Стр \thepage\ из \numpages}{\examdate}
\runningheadrule

\def\width{16}
\def\hauteur{7}
\pointpoints{балл}{баллов}
\hpword{Баллов:}
\hqword{Задача}
\htword{Итог}
\hsword{Оценка:}
\vpword{Баллов:}
\vqword{Задача}
\vtword{Итог}
\vsword{Оценка:}

\begin{document}
	
	\noindent
	\begin{tabular*}{\textwidth}{l @{\extracolsep{\fill}} r @{\extracolsep{6pt}} l}
		\textbf{\class} & \textbf{ФИО:} & \makebox[3in]{\hrulefill}\\
		\textbf{\term} &&\\
		\textbf{\examnum} &&
	\end{tabular*}\\
	\rule[2ex]{\textwidth}{2pt}
	
	
	
	\begin{questions}
		\question[2] Как определить множество $S^*$, сопряженное данному $S \subseteq \mathbb{R}^n$?
		\fillwithdottedlines{2em}
		\question[2] Что такое сопряженный конус?
		\fillwithdottedlines{2em}
		\question[2] Как мы определяем многогранник?
		\fillwithdottedlines{2em}
		\question[2] Найти и изобразить на плоскости множество, сопряженное к выпуклому многограннику:
		$$S = \mathbf{conv} \left\{ (-3,-2), (-1,4), (1,1)\right\}$$
		\fillwithdottedlines{4em}
		\begin{center}
			\vspace{-30pt}
			\begin{tikzpicture}[x=1cm, y=0.7cm, semitransparent]
			\draw[step=1mm, line width=0.1mm, black!30!white] (0,0) grid (\width,\hauteur);
			\draw[step=5mm, line width=0.2mm, black!40!white] (0,0) grid (\width,\hauteur);
			\draw[step=5cm, line width=0.5mm, black!50!white] (0,0) grid (\width,\hauteur);
			\draw[step=1cm, line width=0.3mm, black!90!white] (0,0) grid (\width,\hauteur);
			\end{tikzpicture}
		\end{center}
		
		\question[2] Найти сопряженный конус $S^*$ к конусу $S = \left\{ s \in \mathbb{R}^m \mid s = Ax, x \succeq 0 \right\}$, $A \in \mathbb{R}^{m\times n}, x \in \mathbb{R}^n$
		\fillwithdottedlines{8em}
		
		\begin{center}
			\tiny
			\gradetable[h]
		\end{center}
	\end{questions}



\end{document}
