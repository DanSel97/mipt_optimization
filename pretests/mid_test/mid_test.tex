\documentclass[12pt,addpoints]{exam}
\usepackage[T2A]{fontenc}
\usepackage[utf8]{inputenc}
\usepackage[russian]{babel}

\usepackage[margin=1in]{geometry}
\usepackage{amsmath,amssymb}
\usepackage{multicol}
\usepackage{tikz}

\newcommand{\class}{Оптимизация}
\newcommand{\term}{Осенний семестр 2017- 2018}
\newcommand{\examnum}{Промежуточная контрольная работа}
\newcommand{\examdate}{24/10/2017}
\newcommand{\timelimit}{60 Minutes}

\pagestyle{head}
\firstpageheader{}{}{}
\runningheader{\class}{\examnum\ - Стр \thepage\ из \numpages}{\examdate}
\runningheadrule

\def\width{16}
\def\hauteur{7}
\pointpoints{балл}{балла}
\hpword{Баллов:}
\hqword{Задача}
\htword{Итог}
\hsword{Оценка:}
\vpword{Баллов:}
\vqword{Задача}
\vtword{Итог}
\vsword{Оценка:}

\begin{document}
	
	\noindent
	\begin{tabular*}{\textwidth}{l @{\extracolsep{\fill}} r @{\extracolsep{6pt}} l}
		\textbf{\class} & \textbf{ФИО:} & \makebox[3in]{\hrulefill}\\
		\textbf{\term} &&\\
		\textbf{\examnum} &&
	\end{tabular*}\\
	\rule[2ex]{\textwidth}{2pt}
	
	\begin{center}
		\vspace{-25pt}
		\huge Теоретическая часть
	\end{center}
	
	\begin{questions}
		\begingradingrange{theory}
		\question[3] Что такое выпуклая оболочка множества $S \subseteq \mathbb{R}^n$?
		\fillwithdottedlines{2em}
		\question[3] Как определяется коническая оболочка точек $x_1, \ldots, x_m \in \mathbb{R}^n$
		\fillwithdottedlines{2em}
		\question[3] Что такое относительная внутренность множества?
		\fillwithdottedlines{2em}
		\question[3] Что такое опорная гиперплоскость к множеству?
		\fillwithdottedlines{2em}
		\question[3] Напишите Ваши любимые 3 свойства сопряженных множеств.
		\fillwithdottedlines{2em}
		\question[3] Как определяется двойственный конус?
		\fillwithdottedlines{2em}
		\question[3] Что такое якобиан?
		\fillwithdottedlines{2em}
		\question[3] Пусть есть сложная функция $f(g(x)): x \in \mathbb{R}^n; \;\;\; g: \mathbb{R}^n \to \mathbb{R}^p;\;\;\; f : \mathbb{R}^p \to \mathbb{R}^1$. Как считать $\frac{\partial f \left(g(x)\right)}{\partial x_j}$?
		\fillwithdottedlines{2em}
		\question[3] Сформулируйте дифференциальный критерий выпуклости первого порядка.
		\fillwithdottedlines{2em}
		\question[3] Напишите Ваши любимые различные 3 операции, сохраняющие выпуклость функции многих переменных.
		\fillwithdottedlines{2em}
		
		\endgradingrange{theory}
	


\begin{center}
	%\vspace{-25pt}
	\huge Задачи
\end{center}
%\setcounter{question}{0}
	\begingradingrange{tasks}
	\question[4] Найти выпуклую оболочку множества $\left\{ x,y \in \mathbb{R}^2 \mid x^2 + y^2 \le 1, \;\; xy = 0 \right\}$
	\question[4] Пусть $S = \left\{ x \in \mathbb{R}^3 \mid x_3 \ge x_1^2 + x_2^2 \right\}$. Постройте опорную гиперплоскость к этому множеству, которая Вам наиболее симпатична.
	\question[4] Найти сопряженный конус к конусу положительных полуопределенных матриц.
	\question[4] Посчитать $\nabla f(x)$, если $f(x) = (x - x_0)^T A (x - x_0)$, матрица $A$ при этом симметрична. 
	\question[4] Является ли функция $\lambda_{max}(A)$ выпуклой? Какое ограничение на матрицу нужно при этом наложить? Покажите, что это условие существенно.
	
	\pointpoints{балл}{баллов}
	
	\question[6] Вывести уравнение гиперплоскости, разделяющей множества\\ $S_1 = \left\{ x \in \mathbb{R}^2 \mid x_2(x_1 - 1) \ge 3, x_1 < 1 \right\}$ и $S_2 = \left\{ x \in \mathbb{R}^2 \mid (x_2+4)(x_1+2) \ge 3, x_1 > -2 \right\}$
	\question[6] Найти сопряженное множество к $\{ (x, t) \in \mathbb{R}^{n+1} \mid \| x \| \leq t \}$
	\question[6] Посчитать $\dfrac{\partial f}{\partial A}$, если $f(x) = (x - x_0)^T A (x - x_0)$, матрица $A$ при этом симметрична.
	\question[7] Показать, что функция $f(x) = \prod\limits_{i=1}^n (1 - e^{-x_i})^{\lambda_i}$ вогнута на множестве таких $x: \left\{ x \in \mathbb{R}^n_{++} \mid \sum\limits_{i=1}^n \lambda_i e^{-x_i} \le 1 \right\}$. Здесь $\lambda_1, \ldots, \lambda_n$ - положительные константы.
	\endgradingrange{tasks}
	
	\begin{center}
		\vfill
		\tiny
		\partialgradetable{theory}[h][questions]
		\vfill
		\partialgradetable{tasks}[h][questions]
	\end{center}
\end{questions}


\end{document}
Выяснить при каких значениях параметра $\theta$ гиперплоскость $\Gamma_{p, \beta}: p = (2\theta, \theta, 1), \beta = 3$ является опорной в точке $x_0 = (1,-2,3)$ к многограннику $$, задаваемому системой: 