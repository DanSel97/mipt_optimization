\documentclass[12pt,addpoints]{exam}
\usepackage[T2A]{fontenc}
\usepackage[utf8]{inputenc}
\usepackage[russian]{babel}

\usepackage[margin=1in]{geometry}
\usepackage{amsmath,amssymb}
\usepackage{multicol}
\usepackage{tikz}

\newcommand{\class}{Оптимизация}
\newcommand{\term}{Осенний семестр 2017- 2018}
\newcommand{\examnum}{Тест 4}
\newcommand{\examdate}{1/1/2017}
\newcommand{\timelimit}{60 Minutes}

\pagestyle{head}
\firstpageheader{}{}{}
\runningheader{\class}{\examnum\ - Стр \thepage\ из \numpages}{\examdate}
\runningheadrule

\def\width{16}
\def\hauteur{7}
\pointpoints{балл}{баллов}
\hpword{Баллов:}
\hqword{Задача}
\htword{Итог}
\hsword{Оценка:}
\vpword{Баллов:}
\vqword{Задача}
\vtword{Итог}
\vsword{Оценка:}

\begin{document}
	
	\noindent
	\begin{tabular*}{\textwidth}{l @{\extracolsep{\fill}} r @{\extracolsep{6pt}} l}
		\textbf{\class} & \textbf{ФИО:} & \makebox[3in]{\hrulefill}\\
		\textbf{\term} &&\\
		\textbf{\examnum} &&
	\end{tabular*}\\
	\rule[2ex]{\textwidth}{2pt}
	
	
	
	\begin{questions}
		\question[1] Что такое относительная внутренность множества?
		\fillwithdottedlines{4em}
		\question[1] Как определить сильную отделимость множеств?
		\fillwithdottedlines{6em}
		\question[1] Что такое опорная гиперплоскость к множеству в его граничной точке?
		\fillwithdottedlines{6em}
		\question[2] Пусть S $\subseteq \mathbb{R}^n$ - выпуклое замкнутое множество. Пусть так же имеются точки $\mathbf{y} \in \mathbb{R}^n$ и $\mathbf{\pi} \in S$. Какое неравенство, связывающее $\pi, \mathbf{x}, \mathbf{y}$ должно выполняться, чтобы с уверенностью говорить, что $\pi$ - проекция точки $\mathbf{y}$ на множество $S$?
		\fillwithdottedlines{2em}
		
		\question[2] Написать уравнение касательной к поверхности $F(x) = 0$, где  $F: \mathbb{R}^n \rightarrow \mathbb{R}^1$ в точке $x_0$. 
		\fillwithdottedlines{2em}
		\question[3] Найти $\pi_S (y) = \pi$, если $S = \{x \in \mathbb{R}^n \mid \alpha_i \le x_i \le \beta_i, i = 1, \ldots, n \}$.
		\fillwithdottedlines{10em}
		
		\begin{center}
			\tiny
			\gradetable[h]
		\end{center}
	\end{questions}



\end{document}
